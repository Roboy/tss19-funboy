\chapter{\abstractname}

%TODO: Abstract
More and more modern robots can perform various social functions such as maintaining a sustained conversation in natural languages. However, current Human-Robot Interaction experience in Natural Language Dialogues is still far from perfect. We assume that for better interaction, we need to endow our robots with higher-level functionality – such as humour generation capabilities.
In this thesis, we decided to explore how people would react to robots that can produce humorous remarks in a sustained conversation. Given the current state-of-the-art in Language Models, we conditioned one of them using jokes collected over online sources and proposed a Natural Language Dialogue framework - DARVAH for Natural Language Generation tasks providing humorous responses in a conversation. We implemented the DARVAH framework in the Funboy module of the Ravestate Dialogue System. The module is capable of generating humorous responses, evaluating emotions of conversation partners and validating the generated humorous output accordingly. We used the implementation to carry out a proof-of-concept experiment that revealed how people react to a joking humanoid robot. 
The conducted experiment demonstrated great prospects for improving HRI experience via humour-enabled Dialogue Systems. However, further research and experimental evaluation are necessary to collect more data and unlock the full potential of the DARVAH framework.

\makeatletter
\ifthenelse{\pdf@strcmp{\languagename}{british}=0}
{\renewcommand{\abstractname}{Abstract}}
\makeatother
% !TeX root = ../main.tex
% Add the above to each chapter to make compiling the PDF easier in some editors.

\chapter{Introduction}\label{chapter:introduction}

In Robotics, we could recently observe more and more robots being endowed with various behavioural functions not confined to strictly industrial configurations anymore. Now, robots can talk, recognise faces, hug people, and maintain a sustained conversation. We expect them to be personal helpers integrated into our society, meaning robots need to interact with people directly, communicate clearly, and reason thoroughly. However, even if these robots are sociable and likeable, many people remain wary of them. Our society is not yet ready to embrace robotic technologies in everyday life. Thus, researchers need to investigate the communication between robots and people to answer the question: “How can we help people become friends with robots?” 

Unfortunately, today, people still consider Social Robots nothing more than toys – something akin to Aibo. This notion leads to at least two problems. Firstly, we do not expect Social Robots to appear self-contained and intelligent. When robots act independently to fulfil their social tasks, interlocutors often act unreasonably being astonished at such interactions due to its novelty. Secondly, we may still need humanoid robots for many human-centred tasks. Humanoid robots pose a whole new challenge to Human-Robot Interaction. They look similar to us, but they are strong and do not tire. By making general-tasks interactions easier for robots, we are making it harder for people to cooperate with these robots in a shared environment - humanoid robots may appear hostile to us. 

These two problems usually arise from adopting a pure engineering approach. We have three laws of Robotics and safety rules; therefore, we can make safe robots. We have domain metrics and technological processes; therefore, we can make efficient robots. Nevertheless, this approach is still too mechanical to work well in the intricate world of humans. Our social fabric forms a convoluted and multifaceted system of many sometimes-illogical factors. To rely on someone and to trust them, we have to develop a complex set of psychological conditions based on the interactions we had in the past. A successful social exchange alone is not sufficient to fit into a human-centred environment. All conversation partners have to emulate  at least the feeling of belonging in the group.

What if we create a robot that emulates its belonging in a social circle? Theoretically, it is possible. We could draw inspiration from the Chinese Room Argument. Originally suggested against simplifications in AI definitions, the argument works well for imitating of intellectual capabilities. The concept of this thought experiment is rather simple. Imagine you are sitting in a room. Somebody slides a sheet of paper with a list of questions in Chinese through a slit on the wall. Unfortunately, you do not understand Chinese. However, in the room, there are manuals in English that explain how to convert the given set of Chinese characters into new characters corresponding to the answer. You write a correct answer and pass it back through the slit, effectively faking your Chinese skills. Then the idea is simple: let’s make robots demonstrate fake awareness of the group dynamics and the conversation. This approach seems plausible. Sometimes, even people imitate understanding of topics they do not know. 

Unfortunately, we hit a deeper problem. In theory, the approach works. In practice, robots currently cannot maintain a sufficient context in a sustained conversation. This limitation is again a physical one. Today, we still cannot fit enough computational power into a human-sized robot. Even our mobile assistants cannot boast the ability to understand random questions, having access to data centres and the Internet. Are there other ways to tackle this situation?

To solve the problem, we need to make unorthodox choices. Some roboticists choose to pursue an approach focused on anatomical similarity. Therefore, they try to make a robot which will resemble a human body. They use artificial skin based on elastic materials and soft actuation based on tendons and micro-motors, which help to create more natural movements. Special vocoders help to create soothing voice registers. 

Nonetheless, put together, it produces monsters. Small imperfections seem nightmarish to us. We call this phenomenon the uncanny valley. In aesthetics, the valley denotes a relationship between how much an object resembles a human being and our psychological response to it. Objects that match human physiology but are not humans cause us to feel repugnance. The solution is feasible only in the future, as we improve our embodiment technologies and increase acceptance of humanoid robots bringing us back to the square one.

In contrast, we can try another approach by choosing a principle that brings people together universally. For instance, if we think about music, it does feel like a universal phenomenon that works a bit on a subconscious level. Music makes us feel specific emotions associated only with the current auditory information. During verbal interaction, we can argue that humour may serve such a role. Humour allows us to let go and drop our defences. 

Moreover, it results in collective laughter which is a way to defuse a social situation. The laughter initiates neurophysiological effects such as the release of endorphins which cause euphoric sensations, and reduction in the stress hormone, cortisol. Usually, in human interpersonal interaction, we find it easier to associate with people we can have fun with and a laugh together. Could it maybe work for robots too? The answer is: "not without a challenge”. 

Scientists have been trying to crack this enigma. Natural Language Processing incorporates the field of computational humour; in other words, teaching computers to produce humorous outputs in response to user inputs. The field has existed for several decades. However, there have not been many breakthroughs. The lack of success might be due to two reasons. The first one is a performer problem - until very recently, \acrfull{nlg} technologies have not been advanced enough to produce seemingly random yet semantically relevant responses to verbal inputs. However, robots do not need to be strong-AI capable but need to possess a certain “Swiss-knife” conversational toolkit good enough to just fool their conversational partners for awhile.

The second issue is a recipient problem, which is an issue even for human-generated humour. The reason is that we can never assume beforehand that our spectators or conversation partners will like a specific joke or a comedy style. It is the major complication which stands in the way of computational humour. To solve this problem, robots need to learn to approach people personally. The solution requires compiling interlocutors’ profiles with specific styles of humour tailored to each particular person. To evaluate people’s reactions to the humorous speech, our robots have to gauge the reaction perhaps via visual means, capturing the visual cues of the conversation partner. Thus, we need to personalise Human-Robot Interaction by creating interlocutor’s preference profiles and gauging their preferences individually.

Unfortunately, computational humour is currently not very advanced. Therefore, this project is one of the first steps on the long path to make robots appear more self-contained and intelligent through humour generation capability.

\section{Problem Statement}

The thesis aims to explore how people react to jokes produced by a Neural Language Model conditioned on humour data and delivered by a robot in a verbal interaction - we want to see what is possible with the current level of the necessary technologies. To answer this question, we have to develop a Natural Language Dialogue System module that employs current humour generation and emotion recognition capabilities. Combined together, these two functions can form a feedback loop that not only allows to evaluate the reaction on an emotional level but also to adjust the behaviour of the system and examine how this reaction changes. Using the implementation of the proposed humour generation and evaluation framework, we will conduct the experiment and investigate the obtained results. \par

\section{Thesis Structure}

This section covers the structure of the thesis. 

Chapter 1 introduces the idea of the thesis and the problem it is trying to address. We provide an overview of the theoretical background needed to understand the current state of the art in the field of computational humour and emotion recognition in Chapter 2. It is essential to observe to what extent the previous research in the field brings us closer to the solution. The following chapter provides information about the proposed system design and introduces the \acrfull{darvah} framework, which is the cornerstone of the whole research project (Chapter 3). Specifically, the chapter presents key architecture components and their functions. The chapter also proposes how \acrshort{darvah} should evaluate its interaction with the user.

We describe the technical details of the current system implementation in Chapter 4. There we provide the information about which software modules we built, how the \acrshort{nlg} task works, and what tools we used to create Funboy - our implementation of \acrshort{darvah}. The next chapter (Chapter 5) provides the experiment formulation and its results. There, we explore how people reacted to our robot using the Funboy module to produce humorous responses in the user study. The experiment description provides details regarding its conditions, setup and procedure. Given both the evaluation of the Automatic Emotion Recognition data as well as answers to the self-assessment questionnaires that the participants reported, we discuss the outcome of the user study.

In Chapter 6, we discuss the current challenges that stand in the way of the Funboy project. The experiment revealed several limitations that affect the performance of Funboy and influence the users' perception - we describe the challenges that occurred and possible ways to tackle them in the future. In addition, we propose possible further research since there are a couple of ways to improve the functionality of Funboy in the future.

Chapter 7 concludes the current thesis. 